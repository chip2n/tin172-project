\section*{Contributions}
\subsection*{Andreas Arvidsson}
Most of my work has gone into writing the Interpreter and functions related to the interpretation process.
A huge chunk of this work was focused on writing search functions for the objects (which took a long time to get right) and writing unit tests for every helper function in the interpreter.
The searching functions were relatively straightforward to both implement and write tests for (as mentioned in the report).
I managed to discover some bugs in the development doing this, though, so it was probably worth it.
\newline
\newline
I also worked along Johan and Sebastian to implement the Ambiguity Resolver.
This implementation happened late in the development process, so it forced the group to rethink the goal representation a bit. Specifically, we needed to include the quantifiers in the goals, in order to separate "the" and "any" in the ambiguity resolver.
\newline
\newline
Furthermore, I participed in the group meetings, development sessions and discussion like any good boy would do (except a small period when I was sick), so I have a good grasp on the project as a whole.
\newline
\newline
Finally, I've written the whole Interpreter section of this report, the Introdoction section, roughly half of the Ambiguity Resolver section, some touch ups on the report as a whole, and last but not lest - this whole contribution snippet. Yey!

\subsection*{Sebastian Lagerman}
Together with Robin we created the first draft of the report for the first deadline.
\newline
\newline
I created tests for the validation function which checks the laws of physics.
\newline
\newline
In the introduction of this report I wrote how we decided to interpret the relations which the robot can use.
I was part of writing the ambiguity resolver in this report.
After which I went into details on how the relations heuristics were created and what calculations they perform.
\newline
\newline
In the code I focued mostly on the planner, in which I wrote the check function.
The check function is used in the A* algorithm to check if a goal is satisfied.
I also spent most of my time together with the help of johan to create the heuristics function.
Which also was used by the A* algorithm to calculate how maney steps it could be until it reaches it goal.
This is what took most of my time since I wanted a solid lower bound.
Together with johan and andreas did we create the ambiguity resolver.
I've gone through the code and cleaned up.


\subsection*{Johan Swetz\'en}
\subsection*{Robin Touche}
Worked with Sebastian on the initial report.
\newline
\newline
Implemented the world validation.
How different objects are allowed to interact with each other and making sure no physical laws as defined in the project are violated.
Plus lots of minor code snippets and fixes.
\newline
\newline
Created the \LaTeX-backend for the report.
Organised report and made it follow regulations.
Fixed references with bibtex.
\newline
\newline
Did much code cleanup and documentation.
