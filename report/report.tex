\documentclass[11pt]{article}

\usepackage{color}
\usepackage{eacl2014}
\usepackage{latexsym}
\usepackage{listings}
\usepackage{natbib}
\usepackage{times}
\usepackage{url}

\special{papersize=210mm,297mm}

\bibliographystyle{apalike}

\definecolor{Brown}{cmyk}{0,0.81,1,0.60}
\definecolor{OliveGreen}{cmyk}{0.64,0,0.95,0.40}
\definecolor{CadetBlue}{cmyk}{0.62,0.57,0.23,0}
\definecolor{lightlightgray}{gray}{0.9}

\lstset{
    language=haskell,                       % Code langugage
    basicstyle=\footnotesize\ttfamily,      % Code font, Examples: \footnotesize, \ttfamily
    keywordstyle=\color{OliveGreen},        % Keywords font ('*' = uppercase)
    commentstyle=\color{gray},              % Comments font
    backgroundcolor=\color{lightlightgray}, % Choose background color
    tabsize=2,                              % Default tab size
    captionpos=b,                           % Caption-position = bottom
    breaklines=true,                        % Automatic line breaking?
    breakatwhitespace=false,                % Automatic breaks only at whitespace?
    showspaces=false,                       % Dont make spaces visible
    showtabs=false,                         % Dont make tabls visible
    frame=single,
    framesep=3pt,
    mathescape=true,
    xleftmargin=3pt,
    xrightmargin=3pt
}

\title{Artificial Intelligence Project}
\author{Andreas Arvidsson \\
  {\tt andarv@student.chalmers.se} \\
  Sebastian Lagerman \\
  {\tt seblag@student.chalmers.se} \\
  Johan Swetz\'{e}n \\
  {\tt swetzen@student.chalmers.se} \\
  Robin Touche \\
  {\tt robint@student.chalmers.se} \\}

\date{}

\begin{document}
\maketitle

% 5. Miscellaneous uretionalistics(?)

% 6. Literal use survey

% INCLUDE SOME EXAMPLES OF BOTH GOOD AND BAD SITUATIONS

% not naturel language, but controlled naturel language.
% No programming language terms
% - Software development terms can be used
%   Examples: (FIGURE/FLOWCHART)
% -> utterance -> parse tree -> PDDL goals -A-> PDDL goal -P-> actions
% Pseudo code snippets
% use template provided
% individual contibutions
% 8 pages + appendix/references

\section*{Introduction}
% 1. Introduction
% - Foundation of the world
%   (how spatial relations are interpreted
%    how goals are seperated)
We have been working on an artificial intelligence project which has resulted in an interpreter, ambiguity resolver and planner for an organizing robot simulator.
The robot can react to user commands and move objects in the world accordingly.
In this report, we will describe some implementation details of the various parts of the application itself, as well as our experiences and dilemmas.

\subsection*{Workflow}
The group decided to work with an agile workflow, with regular meetings and development sessions.
With this kind of workflow, we had the ability to handle one utterance at a time.
To avoid as much overhead of working multiple persons on a single code base, we decided to use git for version control.
This helped the group to work in different part of the code without conflicts.

We did not, however, use a Scrum tool like PivotalTracker. If we would have done this, the project might have finished quicker - but unfortunately we realized this a bit too late.
Instead, we mainly used github's issue tracking system for long-lasting bugs (which there weren't many of).

\subsection*{Notation}
In order to simplify the work of both us and the reader, we will use a notation for describing function types which is inspired by the native Haskell notation.
In case the reader is not familiar with this notation - here's an example:

\begin{lstlisting}
funName :: ArgType $\rightarrow$ ArgType2 $\rightarrow$ ReturnType
\end{lstlisting}

The notation starts with the function name ("funName"), which is separated from the rest of the type by a double colon ("::").
Then, the input argument types are separated by arrows ("$\rightarrow$") - with the exception of the last variable, which is always the return type.
So, to write a function which takes a list of integers and returns a bool, we could write:

\begin{lstlisting}
foo :: [Int] $\rightarrow$ Bool
\end{lstlisting}

It is a very concise way to write function types, which is why we've chose to use it. However, this notation will be used sparingly throughout the report to show the most important functions in the various parts of the application.

\subsection*{Relations}
There are a few object relations which the user can provide via the utterance (or command). These are "beside", "left of", "right of", "above", on top", "under" and "inside". To avoid confusion, we will interpret these in the following way:

\begin{description}
  \item[Beside] The assumption here was that the two objects should be placed in
    two columns that are located next to eachother.
  \item[Leftof] We interpreted this to be that the first object would have to be
    placed in one of the columns that are left of the second objects column and
    vice versa.
  \item[Rightof] Just as in \verb|Leftof|, but the two objects are flipped.
  \item[Above] This was interpreted to mean that the two objects would be placed
    in the same column with the first object being above the second one, but
    there could exist objects between them.
  \item[Ontop] We took this to mean that the first object needed to be placed
   strictly on the second object with no objects in between them.
  \item[Under] This was interpreted just as \verb|Above|, but the two objects
    were flipped.
  \item[Inside] We assumed this relation could only be applied to boxes, but
  otherwise it follows the \verb|Ontop| relation.
\end{description}


\section*{Parser}
%Refactored the project and added a test which currently fails // Andreas
%instance Eq Size instead of deriving
The group started of by creating instances for Eq-class for the different types
such as Size, Color and Form. This was done to add the possiblity for the value
any- size, color or form and also to simplify the search process.
%The parser is currently unchanged. We intend to extend it with the negation
%command as mentioned below.


\section*{Interpreter}
The interpreter will be working with the world to find objects that match the
criteria such that the planner only works with the objects id and not the
objects description.

The implementation of the interpreter began by writing a suitable type for it. We knew that the interpreter needs to have access to some kind of world state, which it will search for objects. And the goal of the interpreter is to interpret a user command into a list of possible interpretations for the planner. So, we created the following types:

\begin{lstlisting}[language=haskell]
data State = State { world :: World
                   , holding :: Maybe Id
                   , objects :: Objects }

type World = [[Id]]

data Goal = TakeGoal GoalObject
          | PutGoal Relation GoalObject GoalObject

data GoalObject = Flr | Obj Id
\end{lstlisting}
Using these types, we could write the interpretation type signature as follows:

\begin{lstlisting}
interpret :: State -> Command -> [Goal]
\end{lstlisting}

The implementation of interpret started with the realization that we needed a way to find objects in the world based on their appearance. In order to do this, we started fleshing out simple search functions, so that we can construct PDDL goals that references object IDs. The plan was that if we find multiple matching objects, we can either return one of them, or return an ambiguity error - depending on the user utterance.

\begin{lstlisting}
searchObjects ::
  State ->
  Object ->
  Quantifier ->
  Maybe Location ->
  Either [[(Id,Object)]] [(Id,Object)]

locationHolds :: State -> (Id, Object) -> Location -> Bool
\end{lstlisting}

The searchObjects function takes the world state, an Object (which is a description of the object's appearance), a quantifier (from the utterance) and an optional relative location. It returns either an ambiguity error (Left) bundled with all the possible matches, or a list of all the correctly matched items. In order to handle locations correctly, we needed the locationHolds function, which basically checks if a location is valid for the provided object in the provided world. Both were relativelt straight forward to implement, although it should be noted that a few bugs were discovered during the test writing.


% 3. Ambiguity resolution
% - How are multiple PDDL goals handled? pick first/random? clarification
%   question? return ambiguity error.

\section*{Ambiguity resolver}
In order to resolve any ambiguities from the output of the interpreter, the ambiguity resolver needs the state of the world and, of course, the list of goals themselves.
The type of this function can therefore be written as:
\begin{lstlisting}
resolveAmbiguity :: State $\rightarrow$ [[Goal]] $\rightarrow$ Either AmbiguityError Goal
\end{lstlisting}

We have a list of lists of goals, since there will be one list of goal for each possible interpretation of the user utterance.
The return type of our ambiguity resolver is either an error (in case of ambiguities due to quantifiers or other reasons) or a single goal which will be sent to the planner.

\subsection*{Physical laws}
Once we have received the goals from the input of the main ambiguity resolver function, we need to check all of them if the laws of physics will apply on all of them.
The check is fairly simple.
We iterate over the list of goals, check which type of goal it is (move or take goals), and process them each to remove any impossible goals.
Notable examples that are being removed in this stage are if the user decides to pick up the floor, or if the user tries to place a brick on top of a ball (which violates the physical laws.

\subsection*{Quantifier check}
After which we check if the quantifiers makes any of the goals obsolete. %TODO go inte more detail

% The current ambiguity resolver i very simple. It just checks the number of
% possible interpretations. If we only have one it passes it through, otherwise
% it simply rejects the command.



\section*{Planner}
The planner is basically untouched as of yet. We have been working mostly on
the interpreter and ambiguity resolver. It can currently only partially execute
the take command. It can find the correct column where the requested object
lies but only picks up the top object in that column.



\section*{Extensions}
\begin{itemize}
\item Ambiguity resolution by listing possible different objects, eg. ``You could mean the Yellow Box or the Red Box'', if the user wanted to pick ``the large box'' although there were two of them.
\item Find the shortest solution, measured by the number of pick and drops. This is inherent to the A* algorithm that we are using.
\item Support for differentiating between quantifiers. `a' and `any' means the same, but `the' requires the specification to refer to a single object, or we output an ambiguity error.
\end{itemize}

\section*{Final thoughts}
The project was immensely helpful for our understanding of artificial intelligence for problem solving. It would've been nice if all the groups didn't have exactly the same problem, since we had some cool ideas and watching the same solution in the presentation slots might be a bit boring.

Our project organization could perhaps have been a little better planned with regards to the structure of the code.
Late in the project, when we needed to handle errors in a nice way from the user's perspective, we realized that using monads for this would be a nice solution.
If we'd begun with this realization (which could have been achieved with more discussions and meetings), we'd have saved some time.

Finally, we did not use a Scrum tool like PivotalTracker. If we would have done this, the project might have finished quicker - but unfortunately we realized this a bit too late.
Instead, we mainly used github's issue tracking system for long-lasting bugs (which there weren't many of).

\newpage
\section*{Contributions}
\subsection*{Andreas Arvidsson}
Most of my work has gone into writing the Interpreter and functions related to the interpretation process.
A huge chunk of this work was focused on writing search functions for the objects (which took a long time to get right) and writing unit tests for every helper function in the interpreter.
The searching functions were relatively straightforward to both implement and write tests for (as mentioned in the report).
I managed to discover some bugs in the development doing this, though, so it was probably worth it.
\newline
\newline
I also worked along Johan and Sebastian to implement the Ambiguity Resolver. This implementation happened late in the development process, so it forced the group to rethink the goal representation a bit. Specifically, we needed to include the quantifiers in the goals, in order to separate "the" and "any" in the ambiguity resolver.
\newline
\newline
Furthermore, I participed in the group meetings, development sessions and discussion like any good boy would do, so I have a good grasp on the project as a whole.
\newline
\newline
Finally, I've written this contribution snippet. Yey!

\subsection*{Sebastian Lagerman}
Together with Robin we created the first draft of the report for the first deadline.
\newline
\newline
I created test for the validation function which checks the laws of physics.
\newline
\newline
In this report I wrote the introduction, ambiguity resolver and the heuristics.
\newline
\newline
As of the code I also wrote the checks and heuristics for the A* function and was part of the creation of the ambiguity resolver

\subsection*{Johan Swetz\'en}
\subsection*{Robin Touche}
Implemented the world validation.
How different objects are allowed to interact with each other and making sure no physical laws as defined in the project are violated.
Plus lots of minor code snippets and fixes.

Created the \LaTeX-backend for the report.
Organised report and made it follow regulations.
Fixed references with bibtex.

Did much code cleanup and documentation.


\newpage
\bibliography{references}

\end{document}
