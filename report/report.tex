\documentclass[11pt]{article}

\usepackage{color}
\usepackage{eacl2014}
\usepackage{latexsym}
\usepackage{listings}
\usepackage{natbib}
\usepackage{times}
\usepackage{url}

\special{papersize=210mm,297mm}

\bibliographystyle{apalike}

\definecolor{Brown}{cmyk}{0,0.81,1,0.60}
\definecolor{OliveGreen}{cmyk}{0.64,0,0.95,0.40}
\definecolor{CadetBlue}{cmyk}{0.62,0.57,0.23,0}
\definecolor{lightlightgray}{gray}{0.9}

\lstset{
    language=haskell,                       % Code langugage
    basicstyle=\footnotesize\ttfamily,      % Code font, Examples: \footnotesize, \ttfamily
    keywordstyle=\color{OliveGreen},        % Keywords font ('*' = uppercase)
    commentstyle=\color{gray},              % Comments font
    backgroundcolor=\color{lightlightgray}, % Choose background color
    tabsize=2,                              % Default tab size
    captionpos=b,                           % Caption-position = bottom
    breaklines=true,                        % Automatic line breaking?
    breakatwhitespace=false,                % Automatic breaks only at whitespace?
    showspaces=false,                       % Dont make spaces visible
    showtabs=false,                         % Dont make tabls visible
    frame=single,
    framesep=3pt,
    mathescape=true,
    xleftmargin=3pt,
    xrightmargin=3pt
}

\title{Artificial Intelligence Project}
\author{Andreas Arvidsson \\
  {\tt andarv@student.chalmers.se} \\
  Sebastian Lagerman \\
  {\tt seblag@student.chalmers.se} \\
  Johan Swetz\'{e}n \\
  {\tt swetzen@student.chalmers.se} \\
  Robin Touche \\
  {\tt robint@student.chalmers.se} \\}

\date{}

\begin{document}
\maketitle

% 5. Miscellaneous uretionalistics(?)

% 6. Literal use survey

% INCLUDE SOME EXAMPLES OF BOTH GOOD AND BAD SITUATIONS

% not naturel language, but controlled naturel language.
% No programming language terms
% - Software development terms can be used
%   Examples: (FIGURE/FLOWCHART)
% -> utterance -> parse tree -> PDDL goals -A-> PDDL goal -P-> actions
% Pseudo code snippets
% use template provided
% individual contibutions
% 8 pages + appendix/references

\section*{Introduction}
% 1. Introduction
% - Foundation of the world
%   (how spatial relations are interpreted
%    how goals are seperated)
We have been working on an artificial intelligence project which has resulted in an interpreter, ambiguity resolver and planner for an organizing robot simulator.
The robot can react to user commands and move objects in the world accordingly.
In this report, we will describe some implementation details of the various parts of the application itself, as well as our experiences with the project and problems we encountered during development.

\subsection*{Workflow}
The group decided to work with an agile workflow, with regular meetings and development sessions.
With this kind of workflow, we had the ability to focus on one utterance at a time.
To avoid as much overhead of working multiple persons on a single code base as possible, we decided to use git for version control.
This helped the group to work in different part of the code without conflicts.

\subsection*{Notation}
In order to simplify the work for both us and the reader, we will use a notation for describing function types which is inspired by the notation used in Haskell.
In case the reader is not familiar with this notation - here's an example:

\begin{lstlisting}
funName :: ArgType $\rightarrow$ ArgType2 $\rightarrow$ ReturnType
\end{lstlisting}

The notation starts with the function name (``funName''), which is separated from the rest of the type by a double colon (``::'').
Then, the input argument types are separated by arrows (``$\rightarrow$'') - with the exception of the last variable, which is always the return type.
So, to write a function which takes a list of integers and returns a bool, we could write:

\begin{lstlisting}
foo :: [Int] $\rightarrow$ Bool
\end{lstlisting}

This is a very concise way to write function types, which is why we've chose to use it. However, this notation will be used sparingly throughout the report to show the most important functions in the various parts of the application.

\subsection*{Relations}
There are a few object relations which the user can provide via the utterance (or command). These are ``beside'', ``left of'', ``right of'', ``above'', ``on top'', ``under'' and ``inside''. To avoid confusion, we will interpret these in the following way:

\begin{description}
  \item[Beside] The assumption here was that the two objects should be placed in
    two columns that are located next to eachother.
  \item[Leftof] We interpreted this to be that the first object would have to be
    placed in one of the columns that are left of the second objects column and
    vice versa.
  \item[Rightof] Just as in \verb|Leftof|, but the two objects are flipped.
  \item[Above] This was interpreted to mean that the two objects would be placed
    in the same column with the first object being above the second one, but
    there could exist objects between them.
  \item[Ontop] We took this to mean that the first object needed to be placed
   strictly on the second object with no objects in between them.
  \item[Under] This was interpreted just as \verb|Above|, but the two objects
    were flipped.
  \item[Inside] We assumed this relation could only be applied to boxes, but
  otherwise it follows the \verb|Ontop| relation.
\end{description}

\subsection*{Goals}
There are two goals which cover all basic utterances that are valid. First, we have a goal we chose to call a TakeGoal. This goal initially just contained the object Id that was to be picked up. However, when doing the Ambiguity Resolver we were forced to add a quantifier taken from the utterance as well. This is necessary to differentiate the goals based on the quantifier the user chose.

The second type of goal is the MoveGoal. This contains two object ids (and their quantifiers), and also a relation between them. Using these, we can both handle user movement commands, but also simple put commands (by creating a MoveGoal with the currently held object as the first id).


\section*{Parser}
The input to the application is in the shape of a JSON string containing the world state, the user's utterance and other information that the interpreter and planner needs.
This string is sent from the web interface that was provided for us.
When the application receives the JSON string, it parses it into a bunch of useful data structures which we can later use in various parts of the application.
Much of this code was provided along with the web interface - we've only extended the parser with a few convenience functions for object comparisons.
We also extended the parsing to specific data structures like the world state and an object map which maps object ids to their corresponding descriptions.

\subsection*{State}
The data type representing world state is, in our case, very simple.
It contains the world obtained from the JSON parser, which is just a list of lists where each inner list represents a stack of object ids.
The State data type also contains the object Id of the currently held object and, of course, a map for receiving object descriptions based on their object id. This is all the information we need for the interpreter regarding the world and, as it turns out, also all we need for the Planner and Ambiguity Resolver.

\subsection*{Command}
A command is created from the user's utterance. This can be one of three data types - Take, Put and Move - depending on the utterance.
They contain a description of the objects, and this is used later in the interpreter to convert it to goals for the ambiguity resolver and planner.


\section*{Interpreter}
The interpreter will be working with the world to find objects that match the criteria such that the planner can work with the objects id and not the objects description. Keeping this in mind, we will need a way to find object based on their description. This will be the main focus on the interpreter. A few ambiguity errors can occur here - mainly based on which quantifier is used. This has to be included in the interpretation output, so that the ambiguity resolver can handle it.

The implementation of the interpreter began by writing a suitable type for it.
We knew that the interpreter needs to have access to some kind of world state, which it will search for objects.
And the goal of the interpreter is to interpret a user command into a list of possible interpretations for the planner.
Using these observations, the type would be:

\begin{lstlisting}
interpret :: State -> Command -> Either InterpretationError [Goal]
\end{lstlisting}

We decided to make a monadic interface for the interpreter, which is why the return type looks a bit funky. This was partly because we didn't want to pass the world state around all helper functions, but also to be able to handle possible errors.
So, if the interpreter can't, for example, find a certain entity matching the user command, we can return an error which can be used in other parts of the program to give the user a nice output.

\subsection*{State}
The data type representing world state is, in our case, very simple.
It contains the world obtained from the JSON parser, which is just a list of lists where each inner list represents a stack of object ids.
The State data type also contains the object Id of the currently held object and, of course, a map for receiving object descriptions based on their object id.

\subsection*{Goal}
The end goal of the interpreter is to convert the different user utterance interpretation (from the parser) to a list of different PDDL goals. In order to do this, we need to, as mentioned above, translate the object descriptions to their corresponding id. This is done by running the different search functions. When we receive the result from them, we just create the goals by checking location constraints etc for all the matching objects.

An important addition that we did late in the project was to include the quantifiers in the goals, so that the ambiguity resolver knows how to handle quantity-based ambiguities.

\subsection*{Object search}
The implementation of the interpreter started with the realization that we needed a way to find objects in the world based on their appearance. In order to do this, we started fleshing out simple search functions, so that we can construct PDDL goals that references object IDs as opposed to the object descriptions available from the user command. The plan was that if we find multiple matching objects, we can either return one of them, or return an ambiguity error - depending on the user utterance.

The searchObjects function takes the world state (implicitly via the monadic interface), an object description, a quantifier (from the utterance) and an optional relative location. It returns either an ambiguity error (Left) bundled with all the possible matches, or a list of all the correctly matched items. In order to handle locations correctly, we needed the locationHolds function, which basically checks if a location is valid for the provided object in the provided world. Both were relativelt straight forward to implement, although it should be noted that a few bugs were discovered during the test writing.

\subsection*{Error handling}
The interpreter does not handle any ambiguity errors. This means that it will return a list of goals (coupled with quantifiers). However, there can still be errors. For example, some functions might expect a certain id to exist. If the matching object cannot be found, an interpetation error will be returned.


% 3. Ambiguity resolution
% - How are multiple PDDL goals handled? pick first/random? clarification
%   question? return ambiguity error.

\section*{Ambiguity resolver}
The ambiguity resolver needs the state of the world and all the possible goals that could be interpreted from the input.
Once we have the goals we checks all of them if the laws of physics will apply on al of them.
After which we check if the quantifiers makes any of the goals obsolete.

% The current ambiguity resolver i very simple. It just checks the number of
% possible interpretations. If we only have one it passes it through, otherwise
% it simply rejects the command.



\section*{Planner}
% 4.
Planning
% - Desception of search algorithm
% - Heuristics used
% - Cost function (what is being minimized? - arm movement, number of steps
% - Average call complexity (lower & upper bounds)
% - data structures used to represent world, goal search space
The planner takes the goals that the interpreter and ambiguity resolver have created from an utterance and finds a way to actually fullfill those goals.
The planner uses the A* algorithm to explore possible solutions and shortest solution (smallest number of take/drop steps) based on heuristics as explained below in more detail.

The planner makes sure that only valid solutions (that upholds the given physical laws) will be considered.
However, the algorithm only validates the possible steps taken and avoids impossible states, which means that it assumes the initial world is valid as well.
If not it cannot reach a conclusion.

\subsection*{Heuristics}
We started out by just implementing the ``take'' utterance, so our heuristics were simply the number of objects on top of the sought-after object.
As we continued to add more utterances, that part was left unchanged until we started getting major speed issues.
We long thought that this was an issue with our state being too big, because we saved the whole world in every node of the graph that the A* algorithm was searching through.
After rewriting the graph representation multiple times we realised that the problem was not with the graph, but with the heuristics.
The number of objects above a given object turned out to be a very bad idea, and twice that number is much closer to reality.
This cut our search space down from 7000 nodes (basically bredth-first search) to 40 nodes visited when picking up an object with about five objects above it.

\begin{description}
  \item[Take] Firstly the hueristics checks what the arm is holding at the moment.
    If the objects id matches the one we wish to take then there are no more step needed.
    However if it's another object then we need to remove it from the arm which adds one step.
    Then find the object we are looking for and remove the objects lying above it, this is calculated to two steps for each object.
    Once all that is done then there is only one more step needed to pick up the object we are after.
    One the other hand when we are not holding anything then we do the same calculations, but without adding the one step for clearing the arm.
  \item[Move to floor] A special case was created for puting an object on the floor since the floor otherwise can't be interacted with.
    The hueristics starts of by checking if there are any empty floor spaces.
    After which both cases checks what it is holding and if it's the object we wish to place on the floor.
    If this is true then when there exists a floor space then it's only one step to drop it down, but if it doesn't exist any space then we decided that the number of steps it would take to remove a minimum would be the length of the shortest column multiplied by two since it takes two steps to pick up and drop down an object.
    After which it took three aditional steps, one for puting the object down before clearing away the smallest column and then two steps to pick the object up and drop it down on the floor.
  \item[Ontop] The utterance started of by checking what it was holding.
    If it was the object we wished to place on the second one then we only checked how maney objects were place on it and for each object we count two steps to remove them in addition to three more steps to put down the first object we were holding then picking it up again and placing it on the second object.
    On the other hand if it isn't one of the objects we are suppose to handle then we need to remove it from the arm which takes one step, then we need to clear away all the object on the two objects which takes two steps for each.
    Finally we move the first object onto the second object which also takes to steps.
  \item[Inside] Works exactly like the \verb|Ontop| utterance.
  \item[Beside] This utterance will firstly check if the goal is satisfied and return zero steps.
    Otherwise it checks what is the arm is holding, if it's nothing then it will look up how maney objects are placed on the two objects and take two steps of each object on the smallest pile to then be moved beside the other object which takes two additional steps.
    If the arm is holding something then we check whether it is one of the object and then assume it will only take one step to put it in the column beside the other object.
    Otherwise we need to take a step to free the arm from the object and then go about clearing away the smallest pile of objects on the two objects we wish to place next to eachother as mentioned before.
  \item[LeftOf] Works much like the \verb|Beside| utterance except we don't have to be as strict since we interpret \verb|LeftOf| as any column as long as it's left of the other objects column.
  \item[RightOf] Works like the \verb|LeftOf| utterance except right instead.
  \item[Above] This utterance works as \verb|Ontop| except it checks whether the one of the objects above the second one is the first one.
  \item[Under] Exactly like \verb|Above| except we have flipped the first with the second object.
\end{description}

\subsection*{A*}
The planner uses our own implementation of the A* algorithm which is based on the description by~\cite{apath}.
It takes as input a graph, a heuristics function, a goal-checking function and a starting node.
To represent the nodes, we have chosen to use the world (a list of lists of id strings) and the object that is currently being held.
Our graph is described as a function that takes a node and returns the possible nodes following it, so we do not have to hold an infinite graph in memory.
We use a priority search queue for keeping track of the ``open'' nodes, the ones we have yet to visit, and their estimated cost, i.e.\ path cost + heuristic.
At all times we also save the visited nodes, the cheapest path cost for getting to any node and the parent of each node.
Iterating through the open queue, taking the cheapest node every time, we will eventually end up with a solution that is optimal with respect to the number of pick and drop actions required.

%The planner is basically untouched as of yet.  We have been working mostly on
%the interpreter and ambiguity resolver.  It can currently only partially execute
%the take command.  It can find the correct column where the requested object
%lies but only picks up the top object in that column.



\section*{Extensions}
\begin{itemize}
\iten Ambiguity resolution by listing possible different objects, eg. ``You could mean the Yellow Box or the Red Box'', if the user wanted to pick ``the large box'' although there were two of them.
\item Find the shortest solution, measured by the number of pick and drops. This is inherent to the A* algorithm that we are using.
\item Support for differentiating between quantifiers. `a' and `any' means the same, but `the' requires the specification to refer to a single object, or we output an ambiguity error.
\end{itemize}

\section*{Final thoughts}
The project was immensely helpful for our understanding of artificial intelligence for problem solving. It would've been nice if all the groups didn't have exactly the same problem, since we had some cool ideas and watching the same solution in the presentation slots might be a bit boring.

Our project organization could perhaps have been a little better planned with regards to the structure of the code.
Late in the project, when we needed to handle errors in a nice way from the user's perspective, we realized that using monads for this would be a nice solution.
If we'd begun with this realization (which could have been achieved with more discussions and meetings), we'd have saved some time.

Finally, we did not use a Scrum tool like PivotalTracker. If we would have done this, the project might have finished quicker - but unfortunately we realized this a bit too late.
Instead, we mainly used github's issue tracking system for long-lasting bugs (which there weren't many of).

\newpage
\section*{Contributions}
\subsection*{Andreas Arvidsson}
\subsection*{Sebastian Lagerman}
\subsection*{Johan Swetz\'en}
\subsection*{Robin Touche}


\newpage
\bibliography{references}

\end{document}
