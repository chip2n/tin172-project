\section*{Planner}
% 4.
Planning
% - Desception of search algorithm
% - Heuristics used
% - Cost function (what is being minimized? - arm movement, number of steps
% - Average call complexity (lower & upper bounds)
% - data structures used to represent world, goal search space
The planner takes the goals that the interpreter and ambiguity resolver have created from an utterance and finds a way to actually fullfill those goals.
The planner uses the A* algorithm to explore possible solutions and shortest solution (smallest number of take/drop steps) based on heuristics as explained below in more detail.

The planner makes sure that only valid solutions (that upholds the given physical laws) will be considered.
However, the algorithm only validates the possible steps taken and avoids impossible states, which means that it assumes the initial world is valid as well.
If not it cannot reach a conclusion.

\subsection*{Heuristics}
We started out by just implementing the ``take'' utterance, so our heuristics were simply the number of objects on top of the sought-after object.
As we continued to add more utterances, that part was left unchanged until we started getting major speed issues.
We long thought that this was an issue with our state being too big, because we saved the whole world in every node of the graph that the A* algorithm was searching through.
After rewriting the graph representation multiple times we realised that the problem was not with the graph, but with the heuristics.
The number of objects above a given object turned out to be a very bad idea, and twice that number is much closer to reality.
This cut our search space down from 7000 nodes (basically bredth-first search) to 40 nodes visited when picking up an object with about five objects above it.

\begin{description}
  \item[Take] Firstly the hueristics checks what the arm is holding at the moment.
    If the objects id matches the one we wish to take then there are no more step needed.
    However if it's another object then we need to remove it from the arm which adds one step.
    After which we find the object we are looking for and remove the objects lying above it, this is calculated to take two steps for each object.
    Once all that is done then there is only one more step needed to pick up the object we are after.
    One the other hand when we are not holding anything then we do the same calculations, but without adding the one step for clearing the arm.
  \item[Move to floor] A special case was created for puting an object on the floor since the floor otherwise can't be interacted with.
    The hueristics starts of by checking if there are any empty floor spaces.
    After which both cases checks what it is holding and if it's the object we wish to place on the floor.
    If this is true then when there exists a floor space then it's only one step to drop it down.
    However if it doesn't exist any space then we decided that the number of steps it would take to remove a minimum would be the length of the shortest column multiplied by two since it takes two steps to pick up and drop down an object.
    After which it took three aditional steps, one for puting the object down before clearing away the smallest column and then two steps to pick the object up and drop it down on the floor.
  \item[Ontop] The utterance started of by checking what it was holding.
    If it was the object we wished to place on the second one then we only checked how maney objects were place on it. 
    For each object we count two steps to remove them in addition to three more steps to put down the first object we were holding then picking it up again and placing it on the second object.
    On the other hand if it isn't one of the objects we are suppose to handle then we need to remove it from the arm which takes one step.
    Then we need to clear away all the object on the two objects which takes two steps for each.
    Finally we move the first object onto the second object which also takes two steps.
  \item[Inside] Works exactly like the \verb|Ontop| utterance.
  \item[Beside] This utterance will firstly check if the goal is satisfied and return zero steps.
    Otherwise it checks what is the arm is holding, if it's nothing then it will look up how maney objects are placed on the two objects.
    Then we take two steps of each object on the smallest pile to then be moved beside the other object which takes two additional steps.
    If the arm is holding something then we check whether it is one of the object and then assume it will only take one step to put it in the column beside the other object.
    Otherwise we need to take a step to free the arm from the object and then go about clearing away the smallest pile of objects on the two objects we wish to place next to eachother as mentioned before.
  \item[LeftOf] Works much like the \verb|Beside| utterance except we don't have to be as strict since we interpret \verb|LeftOf| as any column as long as it's left of the other objects column.
  \item[RightOf] Works like the \verb|LeftOf| utterance except right instead.
  \item[Above] This utterance works as \verb|Ontop| except it checks whether the one of the objects above the second one is the first one.
  \item[Under] Exactly like \verb|Above| except we have flipped the first with the second object.
\end{description}

\subsection*{A*}
The planner uses our own implementation of the A* algorithm which is based on the description by~\cite{apath}.
It takes as input a graph, a heuristics function, a goal-checking function and a starting node.
To represent the nodes, we have chosen to use the world (a list of lists of id strings) and the object that is currently being held.
Our graph is described as a function that takes a node and returns the possible nodes following it, so we do not have to hold an infinite graph in memory.
We use a priority search queue for keeping track of the ``open'' nodes, the ones we have yet to visit, and their estimated cost, i.e.\ path cost + heuristic.
At all times we also save the visited nodes, the cheapest path cost for getting to any node and the parent of each node.
Iterating through the open queue, taking the cheapest node every time, we will eventually end up with a solution that is optimal with respect to the number of pick and drop actions required.

%The planner is basically untouched as of yet.  We have been working mostly on
%the interpreter and ambiguity resolver.  It can currently only partially execute
%the take command.  It can find the correct column where the requested object
%lies but only picks up the top object in that column.

